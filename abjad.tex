\documentclass{article}
\usepackage{tenor2015}
\usepackage{times}
\usepackage{ifpdf}
\usepackage[english]{babel}
\usepackage{cite}

\usepackage{inconsolata}
\usepackage{verbatim}

\usepackage{listings}
\lstset{language=Python,
showspaces=false,
showtabs=false,
breaklines=false,
showstringspaces=false,
breakatwhitespace=false,
escapeinside={(*@}{@*)},
keywordstyle=\bfseries,
basicstyle=\scriptsize\ttfamily
}

\def\papertitle{Design Priorities in Abjad:
    A Python API for Formalized Score Control}
\def\firstauthor{Trevor Ba\v{c}a}
\def\secondauthor{Josiah Wolf Oberholtzer}
\def\thirdauthor{Jeffrey Trevi\~{n}o}
\def\fourthauthor{V\'{i}ctor Ad\'{a}n}

\newif\ifpdf
\ifx\pdfoutput\relax
\else
   \ifcase\pdfoutput
      \pdffalse
   \else
      \pdftrue
\fi

\ifpdf % compiling with pdflatex
  \usepackage[pdftex,
    pdftitle={\papertitle},
    pdfauthor={\firstauthor, \secondauthor, \thirdauthor, \fourthauthor},
    bookmarksnumbered, % use section numbers with bookmarks
    pdfstartview=XYZ % start with zoom=100% instead of full screen; 
                     % especially useful if working with a big screen :-)
   ]{hyperref}
  %\pdfcompresslevel=9

  \usepackage[pdftex]{graphicx}
  % declare the path(s) where your graphic files are and their extensions so 
  %you won't have to specify these with every instance of \includegraphics
  \graphicspath{{./figures/}}
  \DeclareGraphicsExtensions{.pdf,.jpeg,.png}

  \usepackage[figure,table]{hypcap}

\else % compiling with latex
  \usepackage[dvips,
    bookmarksnumbered, % use section numbers with bookmarks
    pdfstartview=XYZ % start with zoom=100% instead of full screen
  ]{hyperref}  % hyperrefs are active in the pdf file after conversion
  \usepackage[dvips]{epsfig,graphicx}
  \graphicspath{{./figures/}}
  \DeclareGraphicsExtensions{.eps}
  \usepackage[figure,table]{hypcap}
\fi

\hypersetup{
    colorlinks,
    citecolor=black,
    filecolor=black,
    linkcolor=black,
    urlcolor=black
}

\title{\papertitle}

\fourauthors
  {\firstauthor} {Harvard University \\
    {\tt \href{mailto:trevor.baca@gmail.com}
        {trevor.baca@gmail.com}}}
  {\secondauthor} {Harvard University \\
    {\tt \href{mailto:josiah.oberholtzer@gmail.com}
        {josiah.oberholtzer@gmail.com}}}
  {\thirdauthor} {Carleton College \\
    {\tt \href{mailto:jeffrey.trevino@gmail.com}
        {jeffrey.trevino@gmail.com}}}
  {\fourthauthor} { 
    {\tt \href{mailto:vctradn@gmail.com}
        {vctradn@gmail.com}}}

\begin{document}

\capstartfalse
\maketitle
\capstarttrue

\begin{abstract}
Abjad\footnote{www.projectabjad.org} is an interactive open-source software system designed
to help composers build up complex pieces of music notation in an iterative and
incremental way.  
Abjad is implemented in the Python\footnote{www.python.org}
programming language and architected as an object-oriented collection of
packages, classes and functions. Composers can visualize their work as
publication-quality score at all stages of the compositional process using
Abjad's interface to the LilyPond\footnote{www.lilypond.org} music notation
package. Although the first versions of Abjad were implemented in 1997 and the project website is now visited thousands of times each month, we have never documented
the design priorities that have guided us as we have built the system. In this paper we detail some of the most important principles we have followed in our work
architecting Abjad. The priorities we document here arise in answer to domain-specific questions of music modeling (what are the fundamental elements of music notation? which elements of music notation should be modeled hierarchically? which programming constructs are available to help model the temporal relationships arising between entities in musical score?) as well as in consideration of the ways in which best practices taken from software engineering can apply to the development of a music software system like ours (which programming concepts concerning things like iteration, aggregation and encapsulation make sense to make available to composers? which existing tools to test, document and deploy other open-source projects are available to help develop a music software system like Abjad?). In the sections that follow we discuss the background and motivations that lead us to ask questions like these and then elaborate the design priorities we have arrived at in our ongoing work architecting Abjad.
\end{abstract}

\section{Introduction}\label{sec:introduction}

\begin{comment}
1. Changed "composers, music theorists and musicologists" to just "composers". How do we feel about this?
\end{comment}

Abjad\footnote{www.projectabjad.org} is an interactive software system designed
to help composers build up complex pieces of music notation in an iterative and
incremental way. Abjad is implemented in the Python\footnote{www.python.org}
programming language and architected as an object-oriented collection of
packages, classes and functions. Users can visualize their work as
publication-quality score at all stages of the compositional process using
Abjad's interface to the LilyPond\footnote{www.lilypond.org} music notation
package. Abjad is open-source software available for free download from the
Python Package Index.\footnote{pypi.python.org}

While many environments for both notation and sound production have arisen
within the last twenty years, the following discussion focuses solely on the
production of notation: Abjad enables composers to express both low- and
high-level compositional ideas by extending a widely used programming language
to provide a sufficiently detailed object model of common practice musical
notation. To minimize the restriction of artistic thought's infinite
possibility while maximizing the ability to specify elegantly any arbitrary
symbolic relationship, Abjad does this without prescribing explicit or implicit
models of music or composition: Abjad defines composition narrowly as the act
of creating a document via the encoded aggregation of notational symbols.

\begin{comment}
The current version of Abjad implements 491 public classes and 324 public
functions.
\end{comment}
\section{Background \& Motivations}\label{sec:background}

\subsection{Computational Models of Notation}

Many systems implement detailed models of music explicitly or implicitly, but few of these implement detailed models of notation.\footnote{Computational models of music might entail the representation of higher-level musical entities apparent in the acts of listening and analysis but absent in the symbols of notation themselves, as determined to be creatively exigent. Programming researchers and musical artists have modeled many such extrasymbolic musical entities, such as large-scale form and transition \cite{polansky1991morphological}, \cite{uno1994temporal}, \cite{dobrian1995algorithmic}, \cite{abrams1999higher}, \cite{Yoo1983}, texture \cite{Horenstein:2004kx}, contrapuntal relationships \cite{Boenn:2009oq}, \cite{Acevedo2005}, \cite{Anders:2011kl}, \cite{Balser:1990tg}, \cite{Jones:2000hc}, \cite{uno1994temporal}, \cite{Bell:1995ij}, \cite{farbood2001analysis}, \cite{Cope:2002fv}, \cite{Laurson:2005dz}, \cite{Polansky:2011fu}, \cite{Ebcioglu:1980kl}, harmonic tension and resolution \cite{Melo2003}, \cite{Wiggins1999}, \cite{Foster:1995qa}, melody \cite{Hornel:1993mi}, \cite{Smith:1992pi}, meter \cite{Hamanaka:2005ff}, rhythm \cite{Nauert2007}, \cite{Degazio:1996lh}, \cite{Collins:2003bs}, timbre \cite{Xenakis:1991fu}, \cite{Creasey:1996ye}, \cite{Osaka2004}, temperament \cite{Seymour:2007qo}, \cite{Graf:2006il}, and ornamentation \cite{Ariza:2003zt}, \cite{Chico-Topfer:1998jl}. This work overlaps fruitfully with analysis tasks, and models of listening and cognition can enable novel methods of high-level musical structures and transformations, like dramatic direction, tension, and transition between sections \cite[108]{Collins2009}.} A system that affords a detailed model of music/composition without linking to a sufficiently detailed model of musical notation does not afford automated notation --- sufficiency, however, depends heavily on generative task. For example, if a composer requires an automated notation system to render complex rhythmic ideas that depend typographically on nested tuplets, a system that produces a notation only via a combination of MIDI and quantization must reduce rhythms to a non-hierarchical stream of event times, eliminating the temporally divisive approach of tuplet notation. For many rhythmic applications, though, MIDI suffices. 

Many automated notation systems exist to model musical notation and the act of typographical layout without explicitly affording the computational modeling of music or composition \cite{Smith:1972mw}, \cite{Nienhuys:2003ve}, \cite{Hoos:1998bd}, \cite{hamel1noteability}; many of these systems strongly imply a model of music, such as Gr\'{e}goire for Gregorian chant, Django for guitar tablature, and GOODFEEL for Braille notation \cite{2006}. In this light, feature-rich systems oriented toward classical composers, such as Finale, Sibelius, SCORE, Igor, Berlioz, and Nightingale fit into the mold of systems that model notation with genre as a primary determinant of generative task. Such a system might go so far as to enable a text-based object-oriented model of notation that automates some aspect of an otherwise point-and-click interface, as in the case of Sibelius's ManuScript scripting language \cite{Technology:qc}. 

Many models of musical notation were created for purposes of corpus-based computational musicology. Formats such as Music21, DARM, SMDL, HumDrum, and MuseData model notation with the generative task of searching through a large amount of data \cite{Selfridge-Field:1997ud}. Commercial notation software developers attempted to establish a data interchange standard for optical score recognition (NIFF) \cite{niff1995niff}; since its release in 2004, MusicXML has become a valid interchange format for over 160 applications and maintains a relatively application-agnostic status, as it was designed with the generative task of acting as an interchange format between variously tasked systems \cite{Good:2001if}. (are Igor and Berlioz commercial?)

Notation representations that underly many of these GUI-based systems often go undescribed as computer representations of notation, in favor of discussions about human-computer interaction. For example, Barker and Cantor developed an early model of music notation that underlies a four-oscilloscope GUI and describe their work entirely in terms of user interaction \cite{cantor1971computer}; likewise, discussions of modern commercial notation systems remain similarly oriented, without much awareness or criticism of the underlying computational models of notation. This results in insufficiently detailed models of notation; systems, for example, that provide models only of mensural notation and enable nonmensural notations only as modified instantiations of notations based on measures.

\subsection{The Development of Hierarchical Object Models of Notation}
Many early models of musical notation were not hierarchical, and Lejaran Hiller, in reflecting on decades of automated notation work, identifies the lack of hierarchical organization as a limitation of early work --- although Nick Collins points out that even Hiller's program PHRASE addresses the hierarchical organization of a score up to the level of a phrase, without moving beyond this mid-level musical structure to concerns of large-scale form \cite[108]{Collins2009}. 

There were several object-oriented music environments by 1990 \cite[139]{Polansky:1990fk}, most created in or inspired by the newly popular Smalltalk-80 programming language; while they facilitated the hierarchical modeling of musical abstractions, they omitted or radically simplified the hierarchical nature of common notation. For example, Glen Krasner (Xerox Systems Science Laboratory) created Machine Tongues VIII, a music system that created an object-oriented model of the score/orchestra distinction inherited from Max Mathews' Music N languages, with a simple linear model of ``partOn'' and ``partOff'' command sequences \cite{Krasner:1991uq}, omitting hierarchical organization entirely when the system produced notational output; although subsequent Machine Tongues systems introduced some hierarchical organization via ``note'' objects that inhabited ``event lists,'' systems did not attempt to model the hierarchical detail of all a traditional score's elements. Like Hiller's PHRASE program, Andreas Mahling's CompAss system organized events hierarchically up to the mid-level ``phrase'' level of musical structure \cite{Mahling:1991qf}. These systems extend Smalltalk-80 with interfaces to the MIDI communications protocol: as extensions of Smalltalk, they enabled the user to arbitrarily extend the system with new objects, creating a detailed and hierarchical model of music, usually flattened into a list of noteOn and noteOff commands to be notated or played back via MIDI interface. 

By 1989, Glendon Diener's Nutation system (written in Objective C for the NeXT computer) modeled both musical and notational structure hierarchically through the use of directed graphs \cite{Diener:1991zr}, \cite{Diener:1991ly}, \cite{Diener:1989ve}.

last section: system motivations
	

Linking our design priorities with those of previous systems by describing perceived deficiencies: evaluative priorities for previous systems: sufficiency instead of comprehensiveness, potentially evaluated by sonic result rather than notation, addressability (in HMSL and JMSL). Reintroduce generative task: late 50s through 80s were motivated locally by generative tasks of specific projects until IRCAM's Patchwork approached a generative task of broadly enabling composers; sufficiency determined by comprehensive task.
\section{Notational isomorphism}\label{sec:notational_isomorphism}

Abjad models objects on the page according to common practice notation.

One class per user-creatable notational element: Note, Chord, Rest,
Articulation, Slur, Beam, Tie.

\subsection{Explicit notational modeling}

Abjad models notation explicitly.

Durations: written, assignable.

\texttt{Notes}, \texttt{Chords} and \texttt{Rests} must be instantiated with
assignable written durations. Durational assignability tests whether a duration
can be represented as a power-of-two flag count combined with zero or more
dots. \texttt{7/16} is an assignable duration while \texttt{5/32} and
\texttt{9/8} are not. Non-assignable durations cannot be represented in common
practice notation by a single glyph but only by two or more glyphs tied
together. Abjad will not automatically render a single note with a duration of
\texttt{5/16} as two or more notes tied together. We consider such behavior to
be too implicit. there are too many potentially compositionally valid ways to
render a duration such as \texttt{5/16} into a series of tied assignable
durations: \texttt{1/4 + 1/16}, \texttt{3/16 + 2/16}, \texttt{2/16 + 3/16},
\texttt{1/16 + 1/4}, \texttt{1/8 + 1/8 + 1/16} etc.

Instead we provide affordances for generating tied notes from
non-assignable durations via the \texttt{scoretools.make\_notes()} function.

\begin{lstlisting}
>>> notes = scoretools.make_notes(["c'"], [(5, 16)])
>>> staff = Staff(notes)
>>> print(format(staff))
\new Staff {
    c'4 ~
    c'16
}
\end{lstlisting}

Prolation: diminution vs augmentation.

Likewise, \texttt{Notes} and other leaves cannot be instantiated with durations
with non-power-of-two denominators. Not only are such durations non-assignable
but they also suggest implicit tuplet-derived prolation..

Three notes each having a prolated duration of \texttt{1/12} can be represented
as either three \texttt{1/16} notes in a \texttt{3:4} tuplet or three
\texttt{1/8} notes in a \texttt{3:2} tuplet.

\subsection{Notational aggregation}

We assume notational primitives are the elements of composition.

The act of composition revolves around the aggregation of these primitives into
arbitrarily complex score objects.

Examples: append(), extend(), insert(), attach().

\subsection{Notational visualization}

Abjad makes visualizing notational artifacts simple. Any notational element or
aggregate can be displayed at any time as a PDF via calls to its top-level
\texttt{show()} function.
\input{section_3_score_addressability.tex}
\section{Relationship Modeling}\label{sec:relationship_modeling}

\subsection{Component, Spanner, Indicator}

Some text here.

Components:

\begin{lstlisting}
>>> staff = Staff()
>>> outer_tuplet_one = Tuplet((2, 3), "d''16 ef'8.")
>>> inner_tuplet = Tuplet((4, 5), "cs''16 e'16 d'2")
>>> outer_tuplet_one.append(inner_tuplet)
>>> outer_tuplet_two = Tuplet((4, 5), "d'8 r16 b'16 as'16")
>>> staff.extend([outer_tuplet_one, outer_tuplet_two])
>>> staff.extend("as'8.. fs'32")
>>> show(staff)
\end{lstlisting}

\includegraphics[scale=1.0]{images/section_4_relationship_modeling-1.pdf}


Spanners:

\begin{lstlisting}
>>> leaves = staff.select_leaves()
>>> attach(Tie(), leaves[4:6])
>>> attach(Tie(), leaves[-3:-1])
>>> attach(Slur(), leaves[:2])
>>> attach(Slur(), leaves[2:6])
>>> final_slur = Slur()
>>> attach(final_slur, leaves[7:])
>>> show(staff)
\end{lstlisting}

\includegraphics[scale=1.0]{images/section_4_relationship_modeling-2.pdf}


Indicators:

\begin{lstlisting}
>>> attach(Markup('giocoso', Up).italic(), leaves[0])
>>> attach(Markup('dolce', Up).italic(), leaves[-4])
>>> attach(Dynamic('f'), leaves[0])
>>> attach(Dynamic('p'), leaves[-4])
>>> attach(Articulation('accent'), leaves[0])
>>> attach(Articulation('accent'), leaves[2])
>>> show(staff)
\end{lstlisting}

\includegraphics[scale=1.0]{images/section_4_relationship_modeling-3.pdf}


\subsection{Hierarchical relationships}

Abjad provides concrete object-models for various hierarchical relationships.

Leaves and parentage.

\begin{lstlisting}
>>> staff_leaves = staff.select_leaves()
>>> for leaf in staff_leaves:
...     leaf
... 
Note("d''16")
Note("ef'8.")
Note("cs''16")
Note("e'16")
Note("d'2")
Note("d'8")
Rest('r16')
Note("b'16")
Note("as'16")
Note("as'8..")
Note("fs'32")
\end{lstlisting}


\begin{lstlisting}
>>> tuplet_leaves = inner_tuplet.select_leaves()
>>> for leaf in tuplet_leaves:
...     leaf
... 
Note("cs''16")
Note("e'16")
Note("d'2")
\end{lstlisting}


\begin{lstlisting}
>>> third_note = staff_leaves[2]
>>> third_note
Note("cs''16")
\end{lstlisting}



\begin{lstlisting}
>>> parentage = inspect_(third_note).get_parentage()
>>> parentage.root
<Staff{4}>
\end{lstlisting}


\begin{lstlisting}
>>> parentage.tuplet_depth
2
\end{lstlisting}


\begin{lstlisting}
>>> parentage.prolation
Multiplier(8, 15)
\end{lstlisting}


\subsection{Temporal relationships}

Spanners.

Effective indicators.

Logical ties.

\begin{lstlisting}
>>> spanners = inspect_(leaves[0]).get_spanners(Slur)
>>> first_slur = tuple(spanners)[0]
>>> first_slur.components
Selection(Note("d''16"), Note("ef'8."))
\end{lstlisting}


\begin{lstlisting}
>>> for leaf in leaves:
...     dynamic = inspect_(leaf).get_effective(Dynamic)
...     print(dynamic, leaf)
... 
Dynamic(name='f') d''16
Dynamic(name='f') ef'8.
Dynamic(name='f') cs''16
Dynamic(name='f') e'16
Dynamic(name='f') d'2
Dynamic(name='f') d'8
Dynamic(name='f') r16
Dynamic(name='p') b'16
Dynamic(name='p') as'16
Dynamic(name='p') as'8..
Dynamic(name='p') fs'32
\end{lstlisting}


Logical ties.

\begin{lstlisting}
>>> for logical_tie in iterate(staff).by_logical_tie():
...     logical_tie
... 
LogicalTie(Note("d''16"),)
LogicalTie(Note("ef'8."),)
LogicalTie(Note("cs''16"),)
LogicalTie(Note("e'16"),)
LogicalTie(Note("d'2"), Note("d'8"))
LogicalTie(Rest('r16'),)
LogicalTie(Note("b'16"),)
LogicalTie(Note("as'16"), Note("as'8.."))
LogicalTie(Note("fs'32"),)
\end{lstlisting}


\input{section_5_open_source.tex}

%\begin{acknowledgments}
%You may acknowledge people, projects, 
%funding agencies, etc. 
%which can be included after the second-level heading
%``Acknowledgments'' (with no numbering).
%\end{acknowledgments} 

\bibliography{tenor2015}
\end{document}