\documentclass{article}
\usepackage{tenor2015}
\usepackage{times}
\usepackage{ifpdf}
\usepackage[english]{babel}
\usepackage{cite}

\usepackage{inconsolata}
\usepackage{verbatim}

\usepackage{listings}
\lstset{language=Python,
showspaces=false,
showtabs=false,
breaklines=false,
showstringspaces=false,
breakatwhitespace=false,
escapeinside={(*@}{@*)},
keywordstyle=\bfseries,
basicstyle=\scriptsize\ttfamily
}

\def\papertitle{Design Priorities in Abjad:
    A Python API for Formalized Score Control}
\def\firstauthor{Trevor Ba\v{c}a}
\def\secondauthor{Josiah Wolf Oberholtzer}
\def\thirdauthor{Jeffrey Trevi\~{n}o}
\def\fourthauthor{V\'{i}ctor Ad\'{a}n}

\newif\ifpdf
\ifx\pdfoutput\relax
\else
   \ifcase\pdfoutput
      \pdffalse
   \else
      \pdftrue
\fi

\ifpdf % compiling with pdflatex
  \usepackage[pdftex,
    pdftitle={\papertitle},
    pdfauthor={\firstauthor, \secondauthor, \thirdauthor, \fourthauthor},
    bookmarksnumbered, % use section numbers with bookmarks
    pdfstartview=XYZ % start with zoom=100% instead of full screen; 
                     % especially useful if working with a big screen :-)
   ]{hyperref}
  %\pdfcompresslevel=9

  \usepackage[pdftex]{graphicx}
  % declare the path(s) where your graphic files are and their extensions so 
  %you won't have to specify these with every instance of \includegraphics
  \graphicspath{{./figures/}}
  \DeclareGraphicsExtensions{.pdf,.jpeg,.png}

  \usepackage[figure,table]{hypcap}

\else % compiling with latex
  \usepackage[dvips,
    bookmarksnumbered, % use section numbers with bookmarks
    pdfstartview=XYZ % start with zoom=100% instead of full screen
  ]{hyperref}  % hyperrefs are active in the pdf file after conversion
  \usepackage[dvips]{epsfig,graphicx}
  \graphicspath{{./figures/}}
  \DeclareGraphicsExtensions{.eps}
  \usepackage[figure,table]{hypcap}
\fi

\hypersetup{
    colorlinks,
    citecolor=black,
    filecolor=black,
    linkcolor=black,
    urlcolor=black
}

\title{\papertitle}

\fourauthors
  {\firstauthor} {Harvard University \\
    {\tt \href{mailto:trevor.baca@gmail.com}
        {trevor.baca@gmail.com}}}
  {\secondauthor} {Harvard University \\
    {\tt \href{mailto:josiah.oberholtzer@gmail.com}
        {josiah.oberholtzer@gmail.com}}}
  {\thirdauthor} {Carleton College \\
    {\tt \href{mailto:jeffrey.trevino@gmail.com}
        {jeffrey.trevino@gmail.com}}}
  {\fourthauthor} { 
    {\tt \href{mailto:vctradn@gmail.com}
        {vctradn@gmail.com}}}

\begin{document}

\capstartfalse
\maketitle
\capstarttrue

\begin{abstract}
Abjad\footnote{www.projectabjad.org} is an interactive open-source software
system designed to help composers build up complex pieces of music notation in
an iterative and incremental way.  Abjad is implemented in the
Python\footnote{www.python.org} programming language and architected as an
object-oriented collection of packages, classes and functions. Composers can
visualize their work as publication-quality score at all stages of the
compositional process using Abjad's interface to the
LilyPond\footnote{www.lilypond.org} music notation package. Although the first
versions of Abjad were implemented in 1997 and the project website is now
visited thousands of times each month, we have never documented the design
priorities that have guided us as we have built the system. In this paper we
detail some of the most important principles we have followed in our work
architecting Abjad. The priorities we document here arise in answer to
domain-specific questions of music modeling (what are the fundamental elements
of music notation? which elements of music notation should be modeled
hierarchically? which programming constructs are available to help model the
temporal relationships arising between entities in musical score?) as well as
in consideration of the ways in which best practices taken from software
engineering can apply to the development of a music software system like ours
(which programming concepts concerning things like iteration, aggregation and
encapsulation make sense to make available to composers? which existing tools
to test, document and deploy other open-source projects are available to help
develop a music software system like Abjad?). In the sections that follow we
discuss the background and motivations that lead us to ask questions like these
and then elaborate the design priorities we have arrived at in our ongoing work
architecting Abjad.
\end{abstract}

\input{background.tex}
%\section{Notational reality: objects can be rendered as
notation}\label{sec:notational_representation}

We believe that the objects that composers work with should all theoretically
be viewable as notation. To this end Abjad makes visualizing notational
artifacts simple. Any notational element or element aggregate can be displayed
at any time as a PDF by calling the top-level \texttt{show()} function. The
motivation of the principle is immediate when discussing classes like
\texttt{Note}, \texttt{Rest} and \texttt{Chord} that model printed symbols on
the page. But Abjad invites developers and advanced users to extend the
principle of notational reality to include the abstract relationships in which
music and music composition are rich: users can call \texttt{show()} on
abstract objects like the Abjad \texttt{Duration}\footnote{Also
\texttt{PitchClass}, \texttt{Mode}, \texttt{Scale} and many other classes.}
class to view any duration as a typeset fraction. The decision to notate
durations this way is wholly conventional but increases the extent to which
users can feel confident that they will be able to view the objects they
create. The same is true of many other Abjad classes that model abstract
relationships: \texttt{Mode}, \texttt{Scale}, \texttt{PitchClass} and so on.
%\section{Notational isomorphism}\label{sec:notational_isomorphism}

The set of typographical notation symbols overlaps with the set of
extrasymbolic musical entities, often causing confusion in discourse and
program architecture.\footnote{A ``chord'', for example, might be a vertically
ordered collection of pitch-classes in a harmonic conceit, or it might refer to
the specific arrangement of pitched note heads, stemmed together into a
composite notation symbol that instructs a performer to perform a sound that
consists of several component pitches.} Because of conceptual overlap between
music and notation, it can be difficult to separate a system’s model of
music/composition from its model of notation.

Abjad handles this ambiguity by explicitly modeling symbols on the page
according to common practice notation.

We provide one class per musical/typographic construct, such as \texttt{Note},
\texttt{NoteHead}, \texttt{Chord}, \texttt{Rest}, \texttt{Slur},
\texttt{Articulation} and so forth.

A \texttt{Note} object has one \texttt{NoteHead}, while a \texttt{Chord}
contains a collection of \texttt{NoteHead} objects.

Components:

\begin{lstlisting}
>>> score = Score()
>>> staff_group = StaffGroup()
>>> upper_staff = Staff(name='Upper Staff')
>>> outer_tuplet_one = Tuplet((2, 3), "d''16 ef'8.")
>>> inner_tuplet = Tuplet((4, 5), "cs''16 e'16 d'2")
>>> outer_tuplet_one.append(inner_tuplet)
>>> outer_tuplet_two = Tuplet((4, 5), "d'8 r16 b'16 as'16")
>>> upper_staff.extend([outer_tuplet_one, outer_tuplet_two])
>>> upper_staff.extend("as'8.. fs'32")
>>> lower_staff = Staff(name='Lower Staff')
>>> lower_staff.extend("c8 r8 b8 r8 gf8 r4 cs8")
>>> staff_group.extend([upper_staff, lower_staff])
>>> score.append(staff_group)
>>> show(score)
\end{lstlisting}

\includegraphics[scale=1.0]{images/notational_isomorphism-1.pdf}


Spanners:

\begin{lstlisting}
>>> upper_leaves = upper_staff.select_leaves()
>>> attach(Tie(), upper_leaves[4:6])
>>> attach(Tie(), upper_leaves[-3:-1])
>>> attach(Slur(), upper_leaves[:2])
>>> attach(Slur(), upper_leaves[2:6])
>>> attach(Slur(), upper_leaves[7:])
>>> show(score)
\end{lstlisting}

\includegraphics[scale=1.0]{images/notational_isomorphism-2.pdf}


Indicators:

\begin{lstlisting}
>>> attach(Dynamic('f'), upper_leaves[0])
>>> attach(Dynamic('p'), upper_leaves[-4])
>>> attach(Articulation('accent'), upper_leaves[0])
>>> attach(Articulation('accent'), upper_leaves[2])
\end{lstlisting}


\begin{lstlisting}
>>> lower_leaves = lower_staff.select_leaves()
>>> attach(Clef('bass'), lower_leaves[0])
>>> for note in iterate(lower_staff).by_class(Note):
...     attach(Articulation('staccato'), note)
... 
>>> show(score)
\end{lstlisting}

\includegraphics[scale=1.0]{images/notational_isomorphism-3.pdf}


\subsection{Building up score: iterative aggregation}

Abjad assumes notational primitives are the elements of composition. The act of
composition then revolves around the iterative aggregation of notational
primitives into arbitrarily complex score objects. Abjad affords aggregation
via Python's \emph{mutable sequence} protocol, a collection of instance methods
which allow score components to be appended, extended or inserted into other
container-like score components as though they were lists.

Spanners such as slurs, beams and glissandi and indicators such as
articulations and textual directions can be attached to score components via
the \texttt{attach()} function.

Abjad attempts to be compositionally agnostic. By providing simple and
unambiguous means of gradually aggregating arbitrarily complex score objects,
Abjad encourages users to develop their own personal approach.

\subsection{Modeling notation explicitly}

Abjad models notation explicitly. All notational primitives expressed by Abjad
must conform to the principles of common practice notation. When compositional
inputs cannot be expressed in terms of these principles, Abjad provides
affordances for massaging them into valid notational states.

For example, Abjad expresses the durations of all score components in terms of
rational values -- fractions and integers -- rather than floating point
numbers. Likewise Abjad expresses all pitches in terms of triples of diatonic
note names, accidentals and octave numbers, rather than MIDI numbers or
frequencies. While Abjad provides alternative representations of pitch and
rhythm, as well as affordances for moving between them, the format actually
stored in and used by score components for rendering notation is always the
most notationally-explicit.

%\subsection{Written, assignable and prolated durations}

All \texttt{Note}, \texttt{Chord} and \texttt{Rest} objects in Abjad must be
instantiated with a duration corresponding to the written glyphs on the page --
a \emph{written} duration.

Written durations must be \emph{assignable}, a category we invented to model
durational initialization. Durational assignability describes whether a
duration can be represented as a power-of-two flag count combined with zero or
more dots. \texttt{1/4}, \texttt{3/16} and \texttt{7/16} are assignable
durations while \texttt{5/32}, \texttt{9/8} and \texttt{1/12} are not.

Non-assignable durations cannot be represented in common practice notation by a
single glyph. They require two or more glyphs with assignable durations tied
together, for the score component to be tupletted, or both.

Abjad will not automatically render a single note with a duration of
\texttt{5/16} as two or more notes tied together. We consider such behavior to
be too implicit. There are too many potentially compositionally valid ways to
render a duration such as \texttt{5/16} into a series of tied assignable
durations: \texttt{1/4 + 1/16}, \texttt{3/16 + 2/16}, \texttt{2/16 + 3/16},
\texttt{1/16 + 1/4}, \texttt{1/8 + 1/8 + 1/16} etc. Instead we provide
affordances for generating tied notes from non-assignable durations. One such
affordance is our \texttt{scoretools.make\_notes()} function, which chooses
smart defaults for generating tied glyphs from otherwise un-notateable input.

\begin{lstlisting}
>>> selection = scoretools.make_notes("c'", [(5, 16)])
>>> staff = Staff(selection)
>>> show(staff)
\end{lstlisting}

\includegraphics[scale=1.0]{images/notational_isomorphism-4.pdf}


All score components also have a \emph{prolated} duration - the product of
their written duration and their \emph{prolation}. Prolation is the cumulative
product of all the \emph{multiplier} of every tuplet found in the
\emph{parentage} of a score component. A score component's prolation depends on
its location in the score hierarchy, and is not an inherent property of itself
independent that hierarchy.

Three \texttt{Note} objects each having a prolated duration of \texttt{1/12}
can be represented as either three \texttt{1/16} notes in a \texttt{3:4} tuplet
or as three \texttt{1/8} notes in a \texttt{3:2} tuplet. As all Abjad
\texttt{Note} objects must have an assignable written duration, the three notes
above must have written durations of either \texttt{1/8} or \texttt{1/16}, and
the tuplet must be correspondingly an explicit diminution or augmentation to
provide the desired prolation of \texttt{2/3} or \texttt{4/3}.

\begin{lstlisting}
>>> selection = scoretools.make_notes("c'", [(1, 12)] * 3)
>>> tuplet = selection[0]
>>> show(tuplet)
\end{lstlisting}

\includegraphics[scale=1.0]{images/notational_isomorphism-5.pdf}

\begin{lstlisting}
>>> tuplet.toggle_prolation()
>>> show(tuplet)
\end{lstlisting}

\includegraphics[scale=1.0]{images/notational_isomorphism-6.pdf}


The durational information of any aggregate score object in Abjad is therefore
always explicit and unambiguous with regard to its notational reality.
%\section{Score Addressability}\label{sec:score_addressability}


\begin{comment}
Score Addressability [Trevor]
    - Iteration
        - methods that "get_*"
            Container.select_leaves
            Container.select_notes_and_chords
            IterationAgent.by_class, .by_logical_tie, .by_timeline, .by_vertical_moment, .depth_first
        - efficient and intuitive navigation of the score hierarchy does what mapping does in a functional program
            -containers partake of Python's sequence iterating interface (for loops work)
    - Structural Addressing
        - numeric addressing
        - temporal addressing
        - named addressing
\end{comment}

\subsection{Addressing objects by index}

Abjad allows the numeric addressing of all score components. Abjad score
components are zero-indexed from the start of the container which holds them:
the statement \texttt{staff[0]} addresses the first component contained in
\texttt{staff} while the statement \texttt{staff[1]} addresses the component
after that, and so on. Negative indices address components from the end of the
container which holds them. Python's slice notation may be used to retrieve an
arbitrary number of contiguous components at one time. As an example of the
latter, the statement \texttt{staff[15:25]} selects the ten components in
\texttt{staff} between indices 15 and 25. The conventions of Abjad's numeric
addressing regime follow those of Python's list and tuple interface exactly. 

\subsection{Addressing objects temporally}

Here is some text.

A \emph{timespan} is an object-oriented model of a start/stop offset pair.
Abjad's \texttt{Timespan} class affords users with a variety of tests for
relationships between timespans such as overlap and intersection.

All durated objects in Abjad have a timespan, including all score components
and spanners, allowing them to partake in timespan-based relationship modeling
without regard for hierarchical score structure.

\subsection{Addressing objects by name}

Here is some text.

\begin{lstlisting}
>>> upper_staff = score['Upper Staff']
>>> show(upper_staff)
\end{lstlisting}

\includegraphics[scale=1.0]{images/score_addressability-1.pdf}


\begin{lstlisting}
>>> lower_staff = score['Lower Staff']
>>> show(lower_staff)
\end{lstlisting}

\includegraphics[scale=1.0]{images/score_addressability-2.pdf}


\subsection{Iterating throught score}

Here is some text.

\begin{lstlisting}
>>> for x in score['Upper Staff']:
...     x
... 
Tuplet(Multiplier(2, 3), 'd\'\'16 ef\'8. Tuplet(Multiplier(4, 5), "cs\'\'16 e\'16 d\'2 ~")')
Tuplet(Multiplier(4, 5), "d'8 r16 b'16 as'16 ~")
Note("as'8..")
Note("fs'32")
\end{lstlisting}


\begin{lstlisting}
>>> for component in iterate(score).by_timeline():
...     component
... 
Note("d''16")
Note('c8')
Note("ef'8.")
Rest('r8')
Note("cs''16")
Note("e'16")
Note("d'2")
Note('b8')
Rest('r8')
Note("d'8")
Note('gf8')
Rest('r16')
Rest('r4')
Note("b'16")
Note("as'16")
Note("as'8..")
Note('cs8')
Note("fs'32")
\end{lstlisting}


\begin{lstlisting}
>>> for logical_tie in iterate(score).by_logical_tie(
...     nontrivial=True,
...     pitched=True,
...     ):
...     logical_tie
... 
LogicalTie(Note("d'2"), Note("d'8"))
LogicalTie(Note("as'16"), Note("as'8.."))
\end{lstlisting}


%\section{Relationship Modeling}\label{sec:relationship_modeling}

We provide a variety of concrete object-models of relationships between objects
in a score.


\subsection{Relating object hierarchically}

Abjad provides concrete object-models for various hierarchical relationships.

Leaves and parentage.

\begin{lstlisting}
>>> upper_leaves = upper_staff.select_leaves()
>>> for leaf in upper_leaves:
...     leaf
... 
Note("d''16")
Note("ef'8.")
Note("cs''16")
Note("e'16")
Note("d'2")
Note("d'8")
Rest('r16')
Note("b'16")
Note("as'16")
Note("as'8..")
Note("fs'32")
\end{lstlisting}


\begin{lstlisting}
>>> tuplet_leaves = inner_tuplet.select_leaves()
>>> for leaf in tuplet_leaves:
...     leaf
... 
Note("cs''16")
Note("e'16")
Note("d'2")
\end{lstlisting}


\begin{lstlisting}
>>> third_note = upper_leaves[2]
>>> third_note
Note("cs''16")
\end{lstlisting}


\begin{lstlisting}
>>> parentage = inspect_(third_note).get_parentage()
>>> parentage.root
<Score<<1>>>
\end{lstlisting}


\begin{lstlisting}
>>> parentage.tuplet_depth
2
\end{lstlisting}


\begin{lstlisting}
>>> parentage.prolation
Multiplier(8, 15)
\end{lstlisting}


\subsection{Relating object in time}

Spanners.

Effective indicators.

Logical ties.

\begin{lstlisting}
>>> spanners = inspect_(upper_leaves[0]).get_spanners(Slur)
>>> first_slur = tuple(spanners)[0]
>>> first_slur.components
Selection(Note("d''16"), Note("ef'8."))
\end{lstlisting}


\begin{lstlisting}
>>> for leaf in upper_leaves:
...     dynamic = inspect_(leaf).get_effective(Dynamic)
...     print(dynamic, leaf)
... 
Dynamic(name='f') d''16
Dynamic(name='f') ef'8.
Dynamic(name='f') cs''16
Dynamic(name='f') e'16
Dynamic(name='f') d'2
Dynamic(name='f') d'8
Dynamic(name='f') r16
Dynamic(name='p') b'16
Dynamic(name='p') as'16
Dynamic(name='p') as'8..
Dynamic(name='p') fs'32
\end{lstlisting}


Logical ties.

\begin{lstlisting}
>>> for logical_tie in iterate(upper_staff).by_logical_tie():
...     logical_tie
... 
LogicalTie(Note("d''16"),)
LogicalTie(Note("ef'8."),)
LogicalTie(Note("cs''16"),)
LogicalTie(Note("e'16"),)
LogicalTie(Note("d'2"), Note("d'8"))
LogicalTie(Rest('r16'),)
LogicalTie(Note("b'16"),)
LogicalTie(Note("as'16"), Note("as'8.."))
LogicalTie(Note("fs'32"),)
\end{lstlisting}


\subsection{Relating objects vertically}

\begin{lstlisting}
>>> for moment in iterate(score).by_vertical_moment():
...     moment
... 
VerticalMoment(0, <<2>>)
VerticalMoment(1/24, <<2>>)
VerticalMoment(1/8, <<2>>)
VerticalMoment(1/6, <<2>>)
VerticalMoment(1/5, <<2>>)
VerticalMoment(7/30, <<2>>)
VerticalMoment(1/4, <<2>>)
VerticalMoment(3/8, <<2>>)
VerticalMoment(1/2, <<2>>)
VerticalMoment(3/5, <<2>>)
VerticalMoment(5/8, <<2>>)
VerticalMoment(13/20, <<2>>)
VerticalMoment(7/10, <<2>>)
VerticalMoment(3/4, <<2>>)
VerticalMoment(7/8, <<2>>)
VerticalMoment(31/32, <<2>>)
\end{lstlisting}


%\input{open_source.tex}

\section{NO DSLs}.
we think it's really important to rely as much as possible on a well-known
and well-understood existent programming langauge. no domain-specific languages
(dsls). music people always think they need a dsl. everybody always makes that
mistake. extend existing language. [TREVOR]

\section{universal illustration}
it's important that composers be able to
illustrate all objects they work with as they compose. this design priority is
reflected in the establishment of what we call the illustration protocol. this
priority is driven in large part by our shared insistence on the incredible
power of conventional music notation as a tool for thinking about sound and
time. (we wanna make it possible to illustrate everything all the time because
we notation helps us think about such incredibly complex stuff.) (footnote: we
think it's important that users have access to all the features of the
underlying typesetter.) [TREVOR]

\section{build bottom-up}
it's important for composers to be able to build complex
score bottom-up. input flexibility. you can build up score via iterative
aggregation but you can also just type a lilypond string into a staff. we
afford both types of input and, indeed, two different ways of building up
score. this is where we should cite alan kay's smalltalk paper. ("it's easy to
make a thing.") we make it easy to build up complex score objects through
iterative aggregation. ("we make it easy to build things up.") [JOSIAH]

\section{build top-down}
it's important for composers to be able to build complex
score top-down. we provide factories. factory functions. factory classes. when
we wanna make complex stuff we usually come up with factories. (there may be a
design principle here that initialization is kept simple and we implement in
factories complex ways of aggregating together objects that admit only simple
initialization.) (we generalize many of these things in processes.) there's a
design priority in here somewhere that what we want to do is make it as easy as
possible for people to implement their own factories. what does this mean? it
means that we want to make it as easy as possible for people to implement their
own code that outputs or creates intermediate level structures or materials.
(there's a realization here that people probabaly don't find it interesting to
make a single note. but people can find it totally fascinating to make an
intermediate-level object like a complete phrase of music.) we think it's
important for composers to be able to create their own processes that
generalize these things. with experience, use of the system migrates from
bottom-up to top-down: just look at us! this is what is closest to the work of
beginning to implement one's own system of composition. this is where we make
the case for esthethic-agnositicism. this is also an important point about
extensibility. [JOSIAH]

\section{selection flexability}
we think it's important for composers to be able to
select arbitrary collections of score objects. this is important for a couple
of reasons. first, so that composers can map operations to the entirety of such
a selection at one time. second, there's some type of conceptual benefit to be
had in named selections; the point is that every selection is an ad hoc
intermediate structure created in the midst of the score. we also think it's
important to afford composers the ability to reference score objects in
whatever ways are most natural for a given task. sometimes the most natural
mode of reference is numeric, sometimes by name or sometimes by music-specific
criteria such as the relative times at which different objects appear in the
score. examples follow. we provide concrete object models of vertical
relationships in the score. [TREVOR]

\section{configuration is super fucking important}
we think it's important for
composers to be able to define an operation (where musical or notational) one
time and be able to apply the operation to arbitrarily many objects in the
score at once. of the many ways abjad provides of doing this, the most
important (and most general) is iteration. examples follow. iteration can be
glossed as "define-once, apply-many". "configuration reuse". we think it's
important for composers to be able to configure objects a single time, and then
resuse the configuration as many times as necessary while composing. we also
think it's important to configure complex objects all at once. "templating". we
think it's important for composers to be able to template all objects in the
system, especially the hugely complex objects. [JOSIAH]

\section{encapsulation}
we want everything to be encapsulated as much as
possible. what this comes out to mean is the system is overwhelming
object-oriented (in the proper uses of the term) and that all parts of the
system (whether object-oriented or not) are structured in such a way as to
provide a single interface named according to a uniform set of naming
conventions. [TREVOR]

\section{open-source best practices}
we try to follow the best practices of the open-source community (which has
a number of subpoints). [TREVOR]

\section{conclusion}
we want composers to become programmers. for extremely good
reasons. and the priorities we have detailed here help make the case for why.
[TREVOR]

\bibliography{tenor2015}
\end{document}