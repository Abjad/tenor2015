\section{Notational isomorphism}\label{sec:notational_isomorphism}

Abjad models objects on the page according to common practice notation.

We assume notational primitives are the elements of composition.

\subsection{Explicit notational modeling}

Abjad models notation explicitly. All notational primitives expressed by Abjad
must conform to the principles of common practice notation. When compositional
inputs cannot be expressed in terms of these principles, we provide affordances
for massaging them into valid notational states.

\subsection{Explicit rationals and pitches}

Abjad expresses durations in terms of rational values -- fractions and integers
-- rather than floating point numbers. Likewise Abjad expresses all pitches in
terms of triples of diatonic note names, accidentals and octave numbers, rather
than MIDI numbers or frequencies. While we provide alternative representations
of pitch and rhythm, as well as affordances for moving between them, the format
actually stored in and used by score components for rendering notation is
always the most notationally-explicit.

\subsection{Written, assignable and prolated durations}

All \texttt{Note}, \texttt{Chord} and \texttt{Rest} objects in Abjad must be
instantiated with a duration corresponding to the written glyphs on the page --
a \emph{written} duration.

Written durations must be \emph{assignable}, a category we invented to model
durational initialization. Durational assignability describes whether a
duration can be represented as a power-of-two flag count combined with zero or
more dots. \texttt{1/4}, \texttt{3/16} and \texttt{7/16} are assignable
durations while \texttt{5/32}, \texttt{9/8} and \texttt{1/12} are not.

Non-assignable durations cannot be represented in common practice notation by a
single glyph. They require two or more glyphs with assignable durations tied
together, for the score component to be tupletted, or both.

Abjad will not automatically render a single note with a duration of
\texttt{5/16} as two or more notes tied together. We consider such behavior to
be too implicit. There are too many potentially compositionally valid ways to
render a duration such as \texttt{5/16} into a series of tied assignable
durations: \texttt{1/4 + 1/16}, \texttt{3/16 + 2/16}, \texttt{2/16 + 3/16},
\texttt{1/16 + 1/4}, \texttt{1/8 + 1/8 + 1/16} etc. Instead we provide
affordances for generating tied notes from non-assignable durations. One such
affordance is our \texttt{scoretools.make\_notes()} function.

\begin{lstlisting}
>>> selection = scoretools.make_notes("c'", [(5, 16)])
>>> staff = Staff(selection)
>>> show(staff)\end{lstlisting}

All score components also have a \emph{prolated} duration - the product of
their written duration and their \emph{prolation}. Prolation is the cumulative
product of all the \emph{multiplier} of every tuplet found in the
\emph{parentage} of a score component. A score component's prolation depends on
its location in the score hierarchy, and is not an inherent property of itself
independent that hierarchy.

Three \texttt{Note} objects each having a prolated duration of \texttt{1/12}
can be represented as either three \texttt{1/16} notes in a \texttt{3:4} tuplet
or as three \texttt{1/8} notes in a \texttt{3:2} tuplet. As all Abjad
\texttt{Note} objects must have an assignable written duration, the three notes
above must have written durations of either \texttt{1/8} or \texttt{1/16}, and
the tuplet must be correspondingly an explicit diminution or augmentation to
provide the desired prolation of \texttt{2/3} or \texttt{4/3}.

\begin{lstlisting}
>>> selection = scoretools.make_notes("c'", [(1, 12)] * 3)
>>> tuplet = selection[0]
>>> show(tuplet)
>>> tuplet.toggle_prolation()
>>> show(tuplet)\end{lstlisting}

The durational information of any aggregate score object in Abjad is always
explicit and unambiguous with regard to its notational reality.

\subsection{Notational aggregation}

We consider the act of composition to revolve around the iterative aggregation
of notational primitives into arbitrarily complex score objects.

- append(), extend(), insert()

- attach()

\subsection{Notational visualization}

Abjad makes visualizing notational artifacts simple.

Any notational element or aggregate can be displayed at any time as a PDF via
calls to its top-level \texttt{show()} function.