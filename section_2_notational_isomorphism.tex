\section{Notational isomorphism}\label{sec:notational_isomorphism}

Abjad models objects on the page according to common practice notation.

We assume notational primitives are the elements of composition.

\subsection{Explicit notational modeling}

Abjad models notation explicitly. All notational primitives expressed by Abjad
must conform to the principles of common practice notation. When compositional
inputs cannot be expressed in terms of these principles, we provide affordances
for massaging them into valid notational states.

\subsection{Explicit rationals and pitches}

Abjad expresses durations in terms of rational values - fractions and integers
- rather than floating point numbers.

Likewise all pitches are expressed in terms of triples of diatonic note names,
accidentals and octave numbers, rather than MIDI numbers or frequencies.

While we provide alternative representations of pitch and rhythm, as well as
affordances for moving between them, the format actually stored in and used by
score components for rendering notation is always the most
notationally-explicit.

\subsection{Assignable durations (and therefore ties)}

All \texttt{Note}, \texttt{Chord} and \texttt{Rest} objects in Abjad must be
instantiated with a duration corresponding to the written glyphs on the page -
a \emph{written} duration.

Written durations must be \emph{assignable}, a category we invented to model
durational initialization.

Durational assignability describes whether a duration can be represented as a
power-of-two flag count combined with zero or more dots.

\texttt{1/4}, \texttt{3/16} and \texttt{7/16} are assignable durations while
\texttt{5/32}, \texttt{9/8} and \texttt{1/12} are not.

Non-assignable durations cannot be represented in common practice notation by a
single glyph.

They require two or more glyphs with assignable durations tied together,
for the score component to be tupletted, or both.

Abjad will not automatically render a single note with a duration of
\texttt{5/16} as two or more notes tied together. We consider such behavior to
be too implicit. There are too many potentially compositionally valid ways to
render a duration such as \texttt{5/16} into a series of tied assignable
durations: \texttt{1/4 + 1/16}, \texttt{3/16 + 2/16}, \texttt{2/16 + 3/16},
\texttt{1/16 + 1/4}, \texttt{1/8 + 1/8 + 1/16} etc.

Instead we provide affordances for generating tied notes from
non-assignable durations.

One such affordance is our \texttt{scoretools.make\_notes()} function.

\begin{lstlisting}
>>> notes = scoretools.make_notes(["c'"], [(5, 16)])
>>> staff = Staff(notes)
>>> print(format(staff))
\new Staff {
    c'4 ~
    c'16
}
\end{lstlisting}

\subsection{Explicit tupleting (augmentation and diminution)}



\subsection{Notational aggregation}

The act of composition then revolves around the iterative aggregation of these
primitives into arbitrarily complex score objects.

- append(), extend(), insert()

- attach()

\subsection{Notational visualization}

Abjad makes visualizing notational artifacts simple.

Any notational element or aggregate can be displayed at any time as a PDF via
calls to its top-level \texttt{show()} function.