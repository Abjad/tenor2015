\section{Notational isomorphism}\label{sec:notational_isomorphism}

Abjad models objects on the page according to common practice notation.

One class per user-creatable notational element: Note, Chord, Rest,
Articulation, Slur, Beam, Tie.

\subsection{Explicit notational modeling}

Abjad models notation explicitly.

Durations: written, assignable.

\texttt{Notes}, \texttt{Chords} and \texttt{Rests} must be instantiated with
assignable written durations. Durational assignability tests whether a duration
can be represented as a power-of-two flag count combined with zero or more
dots. \texttt{7/16} is an assignable duration while \texttt{5/32} and
\texttt{9/8} are not. Non-assignable durations cannot be represented in common
practice notation by a single glyph but only by two or more glyphs tied
together. Abjad will not automatically render a single note with a duration of
\texttt{5/16} as two or more notes tied together. We consider such behavior to
be too implicit. there are too many potentially compositionally valid ways to
render a duration such as \texttt{5/16} into a series of tied assignable
durations: \texttt{1/4 + 1/16}, \texttt{3/16 + 2/16}, \texttt{2/16 + 3/16},
\texttt{1/16 + 1/4}, \texttt{1/8 + 1/8 + 1/16} etc.

Instead we provide affordances for generating tied notes from
non-assignable durations via the \texttt{scoretools.make\_notes()} function.

\begin{lstlisting}
>>> notes = scoretools.make_notes(["c'"], [(5, 16)])
>>> staff = Staff(notes)
>>> print(format(staff))
\new Staff {
    c'4 ~
    c'16
}
\end{lstlisting}

Prolation: diminution vs augmentation.

Likewise, \texttt{Notes} and other leaves cannot be instantiated with durations
with non-power-of-two denominators. Not only are such durations non-assignable
but they also suggest implicit tuplet-derived prolation..

Three notes each having a prolated duration of \texttt{1/12} can be represented
as either three \texttt{1/16} notes in a \texttt{3:4} tuplet or three
\texttt{1/8} notes in a \texttt{3:2} tuplet.

\subsection{Notational aggregation}

We assume notational primitives are the elements of composition.

The act of composition revolves around the aggregation of these primitives into
arbitrarily complex score objects.

Examples: append(), extend(), insert(), attach().

\subsection{Notational visualization}

Abjad makes visualizing notational artifacts simple. Any notational element or
aggregate can be displayed at any time as a PDF via calls to its top-level
\texttt{show()} function.