\section{Open Source}\label{sec:open_source}

\subsection{Compositional Agnosticism}

By providing core functionality oriented toward the elements of standard
notation, Abjad strives to remain as agnostic as possible to various
composition techniques. 

\subsection{Extensibility}

To better afford high-level, personal, eccentric composition techniques as
optional tools packages, Abjad provides clear examples of extensibility through
the authors' own included extensions, including: 

\begin{itemize}
    \item labeltools
    \item metertools
    \item quantizationtools
    \item rhythmmakertools
    \item selectortools
    \item sievetools
    \item tonalanalysistools
\end{itemize}

%\subsection{Affording Extension through Project Structure and Interface}

As an open-source project, composers and researchers can contribute changes via
git pull requests. A process of continuous integration and online version
control simplifies this contribution process. 

\subsection{Embeddability}

Abjad is an importable Python library. It can be used in whole or in part as a
component of any Python-compatible system. Abjad has few Python package
dependencies and is not bound to any specific user application or graphic user
interface. These qualities make Abjad an ideal project ideal for embedding in
other software systems.

For example, Abjad supports IPython
Notebook\footnote{http://ipython.org/notebook.html}, a web-based interactive
computational environment combining code execution, text, mathematics, plots
and rich media into a single document. Notational output from Abjad can be
transparently captured and embedded directly into an IPython Notebook which has
loaded Abjad's IPython Notebook extension. Calls to Abjad's \texttt{show()} are
intercepted and the rendered graphics are embedded directly into the Notebook
along with the generating code. This allows scholars to quickly and intuitively
create music texts which can be shared, edited and executed by other IPython
users.

\subsection{Testability}

Text here.

\subsection{Maintainability}

Text here.