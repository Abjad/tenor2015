\section{Notational reality: objects can be rendered as
notation}\label{sec:notational_representation}

We believe that the objects that composers work with should all theoretically
be viewable as notation. To this end Abjad makes visualizing notational
artifacts simple. Any notational element or element aggregate can be displayed
at any time as a PDF by calling the top-level \texttt{show()} function. The
motivation of the principle is immediate when discussing classes like
\texttt{Note}, \texttt{Rest} and \texttt{Chord} that model printed symbols on
the page. But Abjad invites developers and advanced users to extend the
principle of notational reality to include the abstract relationships in which
music and music composition are rich: users can call \texttt{show()} on
abstract objects like the Abjad \texttt{Duration}\footnote{Also
\texttt{PitchClass}, \texttt{Mode}, \texttt{Scale} and many other classes.}
class to view any duration as a typeset fraction. The decision to notate
durations this way is wholly conventional but increases the extent to which
users can feel confident that they will be able to view the objects they
create. The same is true of many other Abjad classes that model abstract
relationships: \texttt{Mode}, \texttt{Scale}, \texttt{PitchClass} and so on.