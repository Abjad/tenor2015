\section{Background \& Motivations}\label{sec:background}

While many environments for both notation and sound production have arisen
within the last twenty years, the following discussion focuses solely on the
production of notation: Abjad enables composers to express both low- and
high-level compositional ideas by extending a widely used programming language
to provide a sufficiently detailed object model of common practice musical
notation. To minimize the restriction of artistic thought's infinite
possibility while maximizing the ability to specify elegantly any arbitrary
symbolic relationship, Abjad does this without prescribing explicit or implicit
models of music or composition: Abjad defines composition narrowly as the act
of creating a document via the encoded aggregation of notational symbols.

\subsection{Generative Task as an Analytic Framework}

Software production exists as an organizationally designed feedback loop
between production values and implementation \cite{Derniame:1999fk}, and it is
possible to understand a system by understanding the purpose for which it was
initially designed, the system's \emph{generative task(s)}. In the analysis of
systems created for use by artists, this priority yields a dilemma instantly,
as analyses that explain a system's affordances with reference to intended
purpose must contend with the creative use of technology by artists: a system's
intended uses might have little or nothing in common with the way in which the
artist finally uses the technology. For this reason, the notion of generative
task is best understood as an explanation for a system's affordances, with the
caveat that a user can nonetheless work against those affordances to use the
system in novel ways. Generative tasks --- informed by the cultural milieu of
software development, economic constraints of software production, and the
aesthetic proclivities of artists participating in development processes ---
constrain software features to enable a limited subset of possible
representations and user interactions.

While composers working traditionally may allow intuition to substitute for
formally defined principles, a computer demands the composer to think formally
about music \cite{Xenakis:1992rq}. Keeping in mind generative task as an
analytical framework, it is broadly useful to bifurcate an automated notation
system's development into the modeling of music and composition, on the one
hand, and the modeling of musical notation, on the other. All systems model
both, to greater or lesser degrees, often engaging in the ambiguous or implicit
modeling of music and composition while focusing more ostensibly on a model of
notation, or focusing on the abstract modeling of music without a considered
link to a model of notation. Due to the intimate link between notation and
musical ideas, it is impossible for a system that models notation to avoid at
least implicitly modeling musical and compositional ideas, and a computational
model of music and composition is an inevitable component of every automated
notation system, even when it exists as an unspoken set of technological
constraints. Generative task explains a given system's balance between
computational models of music/composition and notation by assuming a link
between intended use and system development.

\subsection{Computational Models of Notation}

Many systems implement detailed models of music explicitly or implicitly, but
few of these implement detailed models of notation.\footnote{Computational
models of music might entail the representation of higher-level musical
entities apparent in the acts of listening and analysis but absent in the
symbols of notation themselves, as determined to be creatively exigent.
Programming researchers and musical artists have modeled many such
extrasymbolic musical entities, such as large-scale form and transition
\cite{polansky1991morphological}, \cite{uno1994temporal},
\cite{dobrian1995algorithmic}, \cite{abrams1999higher}, \cite{Yoo1983}, texture
\cite{Horenstein:2004kx}, contrapuntal relationships \cite{Boenn:2009oq},
\cite{Acevedo2005}, \cite{Anders:2011kl}, \cite{Balser:1990tg},
\cite{Jones:2000hc}, \cite{uno1994temporal}, \cite{Bell:1995ij},
\cite{farbood2001analysis}, \cite{Cope:2002fv}, \cite{Laurson:2005dz},
\cite{Polansky:2011fu}, \cite{Ebcioglu:1980kl}, harmonic tension and resolution
\cite{Melo2003}, \cite{Wiggins1999}, \cite{Foster:1995qa}, melody
\cite{Hornel:1993mi}, \cite{Smith:1992pi}, meter \cite{Hamanaka:2005ff}, rhythm
\cite{Nauert2007}, \cite{Degazio:1996lh}, \cite{Collins:2003bs}, timbre
\cite{Xenakis:1991fu}, \cite{Creasey:1996ye}, \cite{Osaka2004}, temperament
\cite{Seymour:2007qo}, \cite{Graf:2006il}, and ornamentation
\cite{Ariza:2003zt}, \cite{Chico-Topfer:1998jl}. This work overlaps fruitfully
with analysis tasks, and models of listening and cognition can enable novel
methods of high-level musical structures and transformations, like dramatic
direction, tension, and transition between sections \cite[108]{Collins2009}.} A
system that affords a detailed model of music/composition without linking to a
sufficiently detailed model of musical notation does not afford automated
notation --- sufficiency, however, depends heavily on generative task. For
example, if a composer requires an automated notation system to render complex
rhythmic ideas that depend typographically on nested tuplets, a system that
produces a notation only via a combination of MIDI and quantization must reduce
rhythms to a non-hierarchical stream of event times, eliminating the temporally
divisive approach of tuplet notation. For many rhythmic applications, though,
MIDI suffices. 

Many automated notation systems exist to model musical notation and the act of
typographical layout without explicitly affording the computational modeling of
music or composition \cite{Smith:1972mw}, \cite{Nienhuys:2003ve},
\cite{Hoos:1998bd}, \cite{hamel1noteability}; many of these systems strongly
imply a model of music, such as Gr\'{e}goire for Gregorian chant, Django for
guitar tablature, and GOODFEEL for Braille notation \cite{2006}, while
feature-rich systems (often oriented toward classical composers) such as
Finale, Sibelius, SCORE, Igor, Berlioz, and Nightingale, present themselves as
relatively more genre-agnostic software tools. Such a system might go so far as
to enable a text-based object-oriented model of notation that automates some
aspect of an otherwise point-and-click interface, as in the case of Sibelius's
ManuScript scripting language \cite{Technology:qc}. 

Many models of musical notation have been designed created for purposes of
corpus-based computational musicology. Formats such as Music21, DARM, SMDL,
HumDrum, and MuseData model notation with the generative task of searching
through a large amount of data \cite{Selfridge-Field:1997ud}. Commercial
notation software developers attempted to establish a data interchange standard
for optical score recognition (NIFF) \cite{niff1995niff}; since its release in
2004, MusicXML has become a valid interchange format for over 160 applications
and maintains a relatively application-agnostic status, as it was designed with
the generative task of acting as an interchange format between variously tasked
systems \cite{Good:2001if}. (are Igor and Berlioz commercial?)

Notation representations that underly many of these GUI-based systems often go
undescribed as computer representations of notation, in favor of discussions
about human-computer interaction. For example, Barker and Cantor developed an
early model of music notation that underlies a four-oscilloscope GUI and
describe their work entirely in terms of user interaction
\cite{cantor1971computer}; likewise, discussions of modern commercial notation
systems remain similarly oriented, without much awareness or criticism of the
underlying computational models of notation. This results in insufficiently
detailed models of notation; systems, for example, that provide models only of
mensural notation and enable nonmensural notations only as modified
instantiations of notations based on measures.

\subsection{The Development of Hierarchical Object Models of Notation}

Many early models of musical notation were not hierarchical, and Lejaran
Hiller, in reflecting on decades of automated notation work, identifies the
lack of hierarchical organization as a limitation of early work --- although
Nick Collins points out that even Hiller's program PHRASE addresses the
hierarchical organization of a score up to the level of a phrase, without
moving beyond this mid-level musical structure to concerns of large-scale form
\cite[108]{Collins2009}. 

There were several object-oriented music environments by 1990
\cite[139]{Polansky:1990fk}, most created in or inspired by the newly popular
Smalltalk-80 programming language; while they facilitated the hierarchical
modeling of musical abstractions, they omitted or radically simplified the
hierarchical nature of common notation. For example, Glen Krasner (Xerox
Systems Science Laboratory) created Machine Tongues VIII, a music system that
created an object-oriented model of the score/orchestra distinction inherited
from Max Mathews' Music N languages, with a simple linear model of ``partOn''
and ``partOff'' command sequences \cite{Krasner:1991uq}, omitting hierarchical
organization entirely when the system produced notational output; although
subsequent Machine Tongues systems introduced some hierarchical organization
via ``note'' objects that inhabited ``event lists,'' systems did not attempt to
model the hierarchical detail of all a traditional score's elements. Like
Hiller's PHRASE program, Andreas Mahling's CompAss system organized events
hierarchically up to the mid-level ``phrase'' level of musical structure
\cite{Mahling:1991qf}. These systems extend Smalltalk-80 with interfaces to the
MIDI communications protocol: as extensions of Smalltalk, they enabled the user
to arbitrarily extend the system with new objects, creating a detailed and
hierarchical model of music, usually flattened into a list of noteOn and
noteOff commands to be notated or played back via MIDI interface. 

By 1989, Glendon Diener's Nutation system (written in Objective C for the NeXT
computer) modeled both musical and notational structure hierarchically through
the use of directed graphs \cite{Diener:1991zr}, \cite{Diener:1991ly},
\cite{Diener:1989ve}.

last section: system motivations

Linking our design priorities with those of previous systems by describing
perceived deficiencies: evaluative priorities for previous systems: sufficiency
instead of comprehensiveness, potentially evaluated by sonic result rather than
notation, addressability (in HMSL and JMSL). Reintroduce generative task: late
50s through 80s were motivated locally by generative tasks of specific projects
until IRCAM's Patchwork approached a generative task of broadly enabling
composers; sufficiency determined by comprehensive task.